\documentclass[12pt,a4paper]{article}
\usepackage{polski}
\usepackage[polish]{babel}
\usepackage{bbm}
\usepackage{bm}
\usepackage[T1]{fontenc}
\usepackage[utf8]{inputenc}
\usepackage[top=2cm, bottom=2cm, left=3cm, right=3cm]{geometry}
\usepackage{url}
\usepackage{graphics}
%\usepackage{graphicx}
\usepackage[pdftex]{graphicx}
\usepackage{float}
\usepackage{amsmath,amsthm}
\usepackage{enumitem}
\usepackage{epstopdf}
%\usepackage{indentfirst}
\usepackage[labelsep=period]{caption}


\setlist{nolistsep}


\AtBeginDocument{% Fragment zmieniający nazwy 'Rysunek' na 'Wykres' i 'Tablica' na 'Tabela'
        \renewcommand{\tablename}{Tabela}
        \renewcommand{\figurename}{Wykres}
}

\makeatletter
\newcommand{\linia}{\rule{\linewidth}{0.4mm}}


\floatstyle{plain}
\newfloat{image}{h!}{lop}
\floatname{image}{Rysunek}



\let \savenumberline \numberline
\def \numberline#1{\savenumberline{#1.}}


\renewcommand*{\@seccntformat}[1]{%
	  \csname the#1\endcsname
		    .\quad
}


\renewcommand{\maketitle}{\begin{titlepage}
		\vspace*{1cm}
    \begin{center}\small
    	Uniwersytet Wrocławski\\
    	Wydział Matematyki i Informatyki\\
    \end{center}
    \vspace{3cm}
    \noindent
    \linia
    \begin{center}
    	\LARGE{\textsc{\@title}}
         \end{center}
     \linia
    \begin{center}
    	\Large{Kosztorys}
         \end{center}
    \vspace{0.5cm}

    \begin{flushright}

    \begin{minipage}{5.5cm}

    	\small Autorzy:

    \normalsize {\@author} \par
    

    \end{minipage}
    \vspace{5cm}

     

     \end{flushright}

    \vspace*{\stretch{6}}

    \begin{center}

    \@date\\

    \end{center}

  \end{titlepage}%

}


\makeatother

\author{Jakub Stępniewicz (\textbf{233217})\\Dorota Suchocka (\textbf{233218})\\ Radosław Warzocha (\textbf{221136})\\Grupa {\bf XVII}}

\title{Do Domu\\ \small{Nawigator kieszonkowy}}


\begin{document}
\maketitle
\tableofcontents
\vspace{5cm}

\newpage
% 		Ok, najtrudniejsze za nami.		%
% 
\section{Wstęp}
Niniejszy dokument zawiera wycenę poszczególnych elementów projektu. W kosztorysie zoztały
uwzględnione zarówno wydatki związane z zakupem elementów jak i pracą zespołu projektowego.

\section{Wycena elementów aplikacji}
	Wycena poszczególnych elementów produktu została przedstawiona w tabeli \ref{wycena}.
	\begin{table}[h]
 \caption{Wycena elementów}
 \begin{center}
	\begin{tabular}{l|l}
	\texttt{Element} & \texttt{Cena}  \\\hline
	Zakup urządzeń mobilnych do testów (tablet + $2 \cdot$telefon) & $5 000,00$ zł \\\\
	Zaprogramowanie połączenia pomiędzy odbiornikiem GPS a satelitą & $1 500,00$ zł \\\\
	Nawiązanie połaczenia z bazą rozkładów MPK & $1 500,00$ zł \\\\
	Wykonanie właściwego oprogramowania & $7 000,00$ zł \\\\
 \hline\\
 Cena całkowita & $15 000,00$ zł
\end{tabular}
\end{center}
 \label{wycena}
\end{table}

\end{document}

