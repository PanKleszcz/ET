\documentclass[12pt,a4paper]{article}
\usepackage{polski}
\usepackage[polish]{babel}
\usepackage{bbm}
\usepackage{bm}
\usepackage[T1]{fontenc}
\usepackage[utf8]{inputenc}
\usepackage[top=2cm, bottom=2cm, left=3cm, right=3cm]{geometry}
\usepackage{url}
\usepackage{graphics}
%\usepackage{graphicx}
\usepackage[pdftex]{graphicx}
\usepackage{float}
\usepackage{amsmath,amsthm}
\usepackage{enumitem}
\usepackage{epstopdf}
%\usepackage{indentfirst}
\usepackage[labelsep=period]{caption}


\setlist{nolistsep}


\AtBeginDocument{% Fragment zmieniający nazwy 'Rysunek' na 'Wykres' i 'Tablica' na 'Tabela'
        \renewcommand{\tablename}{Tabela}
        \renewcommand{\figurename}{Wykres}
}

\makeatletter
\newcommand{\linia}{\rule{\linewidth}{0.4mm}}


\floatstyle{plain}
\newfloat{image}{h!}{lop}
\floatname{image}{Rysunek}



\let \savenumberline \numberline
\def \numberline#1{\savenumberline{#1.}}


\renewcommand*{\@seccntformat}[1]{%
	  \csname the#1\endcsname
		    .\quad
}


\renewcommand{\maketitle}{\begin{titlepage}
		\vspace*{1cm}
    \begin{center}\small
    	Uniwersytet Wrocławski\\
    	Wydział Matematyki i Informatyki\\
    \end{center}
    \vspace{3cm}
    \noindent
    \linia
    \begin{center}
    	\LARGE{\textsc{\@title}}
         \end{center}
     \linia
    \begin{center}
    	\Large{Kosztorys}
         \end{center}
    \vspace{0.5cm}

    \begin{flushright}

    \begin{minipage}{5.5cm}

    	\small Autorzy:

    \normalsize {\@author} \par
    

    \end{minipage}
    \vspace{5cm}

     

     \end{flushright}

    \vspace*{\stretch{6}}

    \begin{center}

    \@date\\

    \end{center}

  \end{titlepage}%

}


\makeatother

\author{Jakub Stępniewicz (\textbf{233217})\\Rafał Maćkowski (\textbf{233170})\\Grupa {\bf I}}

\title{Symulator tramwaju\\ \small{Też możesz być motorniczym}}


\begin{document}
\maketitle
\tableofcontents
\vspace{5cm}
%	\begin{thebibliography}{9}
%	\bibitem{US} tz, W. Hill {\it Us}, Warszawa 2009.
%	\bibitem{MPK} \url{http://www.mpk.wroc.pl/}
%	\bibitem{SK} \url{http://www.skoda.cz/en/products/tramcars/tramcar-19-t/}
%	\bibitem{UE} \url{http://www.unrealengine.com/}
%	\bibitem{GO} \url{http://www.google.pl/#sclient=psy-ab&hl=pl&source=hp&q=%22symulator+tramwaju+skoda+16t%22&pbx=1&oq=%22symulator+tramwaju+skoda+16t}
%	\end{thebibliography}
\newpage
% 		Ok, najtrudniejsze za nami.		%
% 
\section{Wstęp}
Niniejszy dokument zawiera wycenę poszczególnych elementów projektu. W kosztorysie zoztały
uwzględnione zarówno wydatki związane z zakupem elementów jak i pracą zespołu projektowego.

\section{Wycena elementów symulatora}
	Wycena poszczególnych elementów produktu została przedstawiona w tabeli \ref{wycena}.
	\begin{table}[h]
 \caption{Wycena elementów}
 \begin{center}
	\begin{tabular}{l|l}
	\texttt{Element} & \texttt{Cena}  \\\hline
	Budowa kokpitu & $7 000,00$ zł \\\\
	Konstrukcja i oprogramowanie \textbf{konsoli motorniczego} & $3 000,00$ zł \\\\
	Zaprogramowanie interfejsu komunikacyjnego pomiędzy \\konsolą motorniczego a programem symulującym & $1 500,00$ zł \\\\
	Zakup jednostki centralnej (komputer osobisty)\\ odpowiedzialnej za symulację & $3 500,00$ zł \\\\
	Wykonanie właściwego oprogramowania symulacyjnego & $14 000,00$ zł \\\\
    Silnik graficzy \textit{Unreal Engine 3} & $99$\$ \\
 \hline\\
 Cena całkowita & $29 000,00$ zł + $99$\$
\end{tabular}
\end{center}
 \label{wycena}
\end{table}
\section{Opłaty dodatkowe}
Wersja symulatora odpowiadająca powyższej wycenie nie zawiera realistycznie odwzorowanego miasta
Wrocławia, a jedynie podstawową trasę testową. Pozwala ona oczywiście na pełne korzystanie ze
wszystkich modułów symulatora, jednak aby polepszyć realizm symulacji umożliwiamy zamówienie
specjalnej wersji rozszerzonej. Dodatkowe opłaty związane ze stworzeniem scenerii Wrocławskiej
zostały przedstawione w tabeli \ref{grafa}.

\begin{table}[h]
 \caption{Wycena elementów}
 \begin{center}
	\begin{tabular}{l|l}
	\texttt{Element} & \texttt{Cena}  \\\hline
	Przeniesienie układu torowiska do symulatora & $3000$ zł \\\\
	Modelowanie architektury & $6000$ zł \\\\
	Połączenie układu torowego z drogowym & $2000$ zł \\\\
	Zintegrowanie aktualnego rozkładu jazdy z symulatorem & $2000$ zł \\
 \hline\\
 Cena całkowita & $13 000,00$ zł
\end{tabular}
\end{center}
\label{grafa}
\end{table}
\end{document}

