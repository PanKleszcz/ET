\documentclass[12pt,a4paper]{article}
\usepackage{polski}
\usepackage[polish]{babel}
\usepackage{bbm}
\usepackage{bm}
\usepackage[T1]{fontenc}
\usepackage[utf8]{inputenc}
\usepackage[top=2cm, bottom=2cm, left=3cm, right=3cm]{geometry}
\usepackage{url}
\usepackage{graphics}
%\usepackage{graphicx}
\usepackage[pdftex]{graphicx}
\usepackage{float}
\usepackage{amsmath,amsthm}
\usepackage{enumitem}
\usepackage{epstopdf}
%\usepackage{indentfirst}
\usepackage[labelsep=period]{caption}
\setlist{nolistsep} 

\AtBeginDocument{% Fragment zmieniający nazwy 'Rysunek' na 'Wykres' i 'Tablica' na 'Tabela'
        \renewcommand{\tablename}{Tabela}
        \renewcommand{\figurename}{Wykres}
}

\makeatletter
\newcommand{\linia}{\rule{\linewidth}{0.4mm}}


\floatstyle{plain}
\newfloat{image}{h!}{lop}
\floatname{image}{Rysunek}



\let \savenumberline \numberline
\def \numberline#1{\savenumberline{#1.}}


\renewcommand*{\@seccntformat}[1]{%
	  \csname the#1\endcsname
		    .\quad
}


\renewcommand{\maketitle}{\begin{titlepage}
		\vspace*{1cm}
    \begin{center}\small
    	Uniwersytet Wrocławski\\
    	Wydział Matematyki i Informatyki\\
    \end{center}
    \vspace{3cm}
    \noindent
    \linia
    \begin{center}
    	\LARGE{\textsc{\@title}}
         \end{center}
     \linia
    \begin{center}
    	\Large{Słownik pojęć}
         \end{center}
    \vspace{0.5cm}

    \begin{flushright}

    \begin{minipage}{5.5cm}

    	\small Autorzy:

    \normalsize {\@author} \par
    

    \end{minipage}
    \vspace{5cm}

     

     \end{flushright}

    \vspace*{\stretch{6}}

    \begin{center}

    \@date\\

    \end{center}

  \end{titlepage}%

}


\makeatother

\author{Jakub Stępniewicz (\textbf{233217})\\Rafał Maćkowski (\textbf{233170})\\Grupa {\bf I}}

\title{Symulator tramwaju\\ \small{Też możesz być motorniczym}}


\begin{document}
\maketitle
\tableofcontents
\vspace{5cm}
%	\begin{thebibliography}{9}
%	\bibitem{US} tz, W. Hill {\it Us}, Warszawa 2009.
%	\bibitem{MPK} \url{http://www.mpk.wroc.pl/}
%	\bibitem{SK} \url{http://www.skoda.cz/en/products/tramcars/tramcar-19-t/}
%	\bibitem{UE} \url{http://www.unrealengine.com/}
%	\bibitem{GO} \url{http://www.google.pl/#sclient=psy-ab&hl=pl&source=hp&q=%22symulator+tramwaju+skoda+16t%22&pbx=1&oq=%22symulator+tramwaju+skoda+16t}
%	\end{thebibliography}
\newpage
% 		Ok, najtrudniejsze za nami.		%
% 
\section{Wstęp}
Dokument zawiera objaśnienia trudniejszych pojęć użytych w dokumentacji.

\section{Słownik}

{\bf czuwak aktywny} --– urządzenie sprawdzające obecność oraz czujność {\it motorniczego}. W
przypadku tramwaju {\it Škoda 16T} ma on postać dwóch przycisków wmontowanych w podłogę. Obydwa
muszą być wciśnięte jednocześnie, aby umożliwić ruch {\it jednostki trakcyjnej}.\\\\
{\bf hamulec bębnowy} --– hamulec oparty na zasadzie tarcia, umożliwia postój tramwaju po jego
zatrzymaniu się, kiedy to nie jest już możliwe używanie {\it hamulca elektrodynamicznego}. Jest
także wykorzystywany podczas hamowania awaryjnego.\\\\
{\bf hamulec elektrodynamiczny} --– hamulec wykorzystujący zasadę indukcji elektromagnetycznej.
Hamowanie elektrodynamiczne polega na przełączeniu {\it silników trakcyjnych} w tryb generatora
prądu elektrycznego. Opór przez nie powodowany wykorzystywany jest następnie do zatrzymania pojazdu.
Ten sposób hamowania nie działa jednak przy niskich prędkościach, kiedy to konieczne jest
zastosowanie {\it hamulca bębnowego}. Używanie hamulca elektrodynamicznego pozwala na odzyskanie
większości energii hamowania.\\\\
{\bf interfejs komunikacyjny} --– specjalne oprogramowanie pozwalające na komunikowanie się pomiędzy
modułami symulatora, a także umożliwające na komunikację człowiek-komputer.\\\\
{\bf jednostka trakcyjna} --– autonomiczny pojazd szynowy wyposażony we własne źródło napędu (silnik).
Tu: tramwaj.\\\\
{\bf kokpit} --– element tramwaju pozwalający na jego sterowanie. Składa się z licznych przełączników,
wyświetlaczy oraz nastawników umożliwiających bieżącą kontrolę stanu i parametrów {\it jednostki
trakcyjnej}.\\\\
{\bf laminat poliestrowo-szklany} --– wytrzymałe, łatwe do kształtowania tworzywo sztuczne. Składa się z licznych
warstw mat wykonanych z włókna szklanego połączonych żywicą epoksydową i pokrytych {\it
żelkotem}.\\\\
{\bf motorniczy} --– osoba kierująca tramwajem, odpowiedzialna za komfort, czas i bezpieczeństwo
podróży.\\\\
{\bf pantograf} --– element tramwaju pozwalający na pobieranie napięcia elektrycznego z {\it sieci 
trakcyjnej}.\\\\
{\bf pojazd trakcyjny} --– patrz: {\it jednostka trakcyjna}.\\\\
{\bf programacja drzwi} --– sposób kontroli drzwi polegający na selektywnym wyborze otwieranych drzwi
przez pasażerów. W momencie włączenia programacji otworzą się tylko te drzwi, przy
których pasażerowie wcisnęli odpowiedni przycisk (zlokalizowany zwykle w bezpośrednim sąsiedztwie
	drzwi).\\\\
{\bf przepustnica} --– element kokpitu, dźwignia pozwalająca na sterowanie prędkością {\it jednostki
trakcyjnej}. W przypadku tramwaju {\it Škoda 16T} w jednej dźwigni zintegrowana jest zarówno
kontrola {\it silników trakcyjnych} jak i hamulców {\it bębnowych} oraz {\it
elektrodynamicznych}.\\\\
{\bf przerywnik trakcyjny} --– element {\it sieci trakcyjnej} pozwalający na łącznie dwóch różnych
sektorów zasilających. Składa się z kawałka izolatora umieszczonego bezpośrednio na łączeniu
sektorów. Bardzo ważne jest, aby w momencie przejazdu przez przerywnik nie pobierać napięcia z
sieci, gdyż może to spowodować powstanie łuku elektrycznego, którego wysoka temperatura przyspawa
pantograw do sieci trakcyjnej powodując tym samym jej zerwanie.\\\\
{\bf radiotelefon} --– urządzenie komunikacyjne wykorzystujące fale radiowe jako medium komunikacji.\\\\
{\bf sieć trakcyjna} --– ogólne określenie na sieć kabli, którymi doprowadzany jest prąd do jednostek
trakcyjnych. W przypadku tramwajów Wrocławskich napięcie sieci trakcyjnej wynosi $800V$.\\\\
{\bf silnik trakcyjny} --– źródło napędu {\it jednostki trakcyjnej}. Elektryczny silnik, który
zamienia energie elektryczną pobraną z {\it sieci trakcyjnej} na ruch postępowy tramwaju.\\\\
{\bf symulator} --– (tu: symulator tramwaju) oprogramowanie imitujące rzeczywistość, pozwalające na 
wykorzystanie systemu komputerowego w celu jak najwierniejszego odzwierciedlenia warunków miejsca
pracy {\it motorniczego}, pozwalające na wykonywanie wszystkich (również niebezpiecznych) manewrów 
{\it jednostki trakcyjnej}\\\\
{\bf układ trakcyjny} --– zespół {\it jednostek trakcyjnych}.\\\\
{\bf żelkot} --– zewnętrzna warstwa {\it laminatu poliestrowo-szklanego}, nadająca mu
estetyczny wygląd, kolor, trwałość i odporność na działanie czynników
atmosferycznych (woda, promieniowanie UV).


\end{document}

