\documentclass[12pt,a4paper]{article}
\usepackage{polski}
\usepackage[polish]{babel}
\usepackage{bbm}
\usepackage{bm}
\usepackage[T1]{fontenc}
\usepackage[utf8]{inputenc}
\usepackage[top=2cm, bottom=2cm, left=3cm, right=3cm]{geometry}
\usepackage{url}
\usepackage{graphics}
%\usepackage{graphicx}
\usepackage[pdftex]{graphicx}
\usepackage{float}
\usepackage{amsmath,amsthm}
\usepackage{enumitem}
\usepackage{epstopdf}
%\usepackage{indentfirst}
\usepackage[labelsep=period]{caption}
\setlist{nolistsep}


\AtBeginDocument{% Fragment zmieniający nazwy 'Rysunek' na 'Wykres' i 'Tablica' na 'Tabela'
        \renewcommand{\tablename}{Tabela}
        \renewcommand{\figurename}{Wykres}
}

\makeatletter
\newcommand{\linia}{\rule{\linewidth}{0.4mm}}


\floatstyle{plain}
\newfloat{image}{h!}{lop}
\floatname{image}{Rysunek}



\let \savenumberline \numberline
\def \numberline#1{\savenumberline{#1.}}


\renewcommand*{\@seccntformat}[1]{%
	  \csname the#1\endcsname
		    .\quad
}


\renewcommand{\maketitle}{\begin{titlepage}
		\vspace*{1cm}
    \begin{center}\small
    	Uniwersytet Wrocławski\\
    	Wydział Matematyki i Informatyki\\
    \end{center}
    \vspace{3cm}
    \noindent
    \linia
    \begin{center}
    	\LARGE{\textsc{\@title}}
         \end{center}
     \linia
    \begin{center}
    	\Large{Narzędzia i sposoby realizacji}
         \end{center}
    \vspace{0.5cm}

    \begin{flushright}

    \begin{minipage}{5.5cm}

    	\small Autorzy:

    \normalsize {\@author} \par
    

    \end{minipage}
    \vspace{5cm}

     

     \end{flushright}

    \vspace*{\stretch{6}}

    \begin{center}

    \@date\\

    \end{center}

  \end{titlepage}%

}


\makeatother

\author{Jakub Stępniewicz (\textbf{233217})\\Rafał Maćkowski (\textbf{233170})\\Grupa {\bf I}}

\title{Symulator tramwaju\\ \small{Też możesz być motorniczym}}


\begin{document}
\maketitle
\tableofcontents
\vspace{5cm}
%	\begin{thebibliography}{9}
%	\bibitem{US} tz, W. Hill {\it Us}, Warszawa 2009.
%	\bibitem{MPK} \url{http://www.mpk.wroc.pl/}
%	\bibitem{SK} \url{http://www.skoda.cz/en/products/tramcars/tramcar-19-t/}
%	\bibitem{UE} \url{http://www.unrealengine.com/}
%	\bibitem{GO} \url{http://www.google.pl/#sclient=psy-ab&hl=pl&source=hp&q=%22symulator+tramwaju+skoda+16t%22&pbx=1&oq=%22symulator+tramwaju+skoda+16t}
%	\end{thebibliography}
\newpage
% 		Ok, najtrudniejsze za nami.		%
% 
\section{Budowa kokpitu tramwaju {\it Škoda 16T}}
Jednym z głównych atutów symulatora jest wiernie odwzorowany kokpit trawmaju {\it Škoda 16T}. Budowa
jednak jest przedsięwzięciem bardzo złożonym i wymagającym. Można ją podzielić na kilka etapów.
\subsection{Stelaż}
Szkielet kokpitu zostanie wykonany z kątowników aluminiwych, gdyż umożliwiają one łatwe uzyskanie
sztywności całej konstrukcji, przy nieskomplikownym sposobie montażu. Duża wytrzymałość stelażu
ułatwi umieszczenie pozostałych elementów. Dodatkowo wykonany z aluminium stelaż będzie miał
niewielką masę, co ułatwi ewentualne przemieszczanie kokpitu.

\subsection{Laminatowa powłoka}
Następnie stelaż zostanie pokryty specjalnie do tego celu przygotowaną powłoką z {\it laminatu poliestrowo-szklanego}.
Elementy będą przygotowane metodą łączenia wielu warstw mat szklanych przy użyciu specjalnego
epoksydowego spoiwa. Następnie zostaną one przykręcone do stelażu i pokryte {\it żelkotem} w
odpowiednim kolorze.

\subsection{Przyciski i przełączniki}
Przy użyciu szlifierki zostaną wycięte z aluminium wszystkie metalowe elementy kokpitu, takie jak
czuwaki, czy kolumna przepustnicy. Elementy plastikowe takie jak guziki i przełączniki zostaną
wykonane za pomocą drukarki trójwymiarowej z półprzezroczystego PCV. Zostaną one następnie
umieszczone w odpowiednich miejscach laminatowego stelażu. 

\subsection{Elektronika i wyświetlacze}
Ostatnią częśćią budowy kokpitu jest zamontowanie układów odpowiadających za logikę działania
wszystkich systemów tramwaju. Są to:
\begin{itemize}
\item Kontrolery oświetlenia i podświetlenia kokpitu,
\item układy sterujące zachowaniem się systemów w kokpicie,
\item sterowniki drzwi, wycieraczek, ogrzewania,
\item układy odpowiedzialne za kontrolę zasilania,
\item wielofunkcyjny wyświetlacz aktualnego stanu tramwaju,
\item przepustnica sterująca silnikami trakcyjnymi oraz hamulcami,
\item symulator systemu gps oraz radiotelefonu.
\end{itemize}
Wszystkie te części będą połączone ze specjalnie do tego celu zaprogramowanym mikrokontrolerem, któy
będzie się następnie komunikował z właściwym symulatorem w celu aktualizacji stanu tramwaju.

\section{Implementacja symulatora}
Symulator będzie implementowany w języku {\it C++}, w standardzie {\it GNU99}. Ponieważ sam C++
nie zawiera narzędzi pozwalających w prosty sposób wykonywać interfejsy graficzne, z programem
został połączony silnik {\it Unreal Engine 3}.

\subsection{Fizyka i model tramwaju}
W pełni odwzorowane zostaną następujące elementy tramwaju {\it Škoda 16T}:
\begin{itemize}
\item Model ruchu tramwaju typu włączając:
	\begin{itemize}
	\item system hamowania elektrodynamiczniego,
	\item system hamowania konwencjonalnego,
	\item system odzyskiwania energii hamowania,
	\item działanie silników trakcyjnych,
	\item awarie poszczególnych systemów napędowych,
	\end{itemize}
\item model zasilania trakcyjnego, w tym:
	\begin{itemize}
	\item różne sektory sieci trakcyjnej,
	\item {\it przerywniki},
	\item przepięcia i spadki napięcia,
	\item zachowanie się {\it pantografów} w rożnych warunkach pogodowych,
	\item system zasilania akumalotorowego,
	\end{itemize}
\item działanie systemów logicznych tramwaju:
	\begin{itemize}
	\item kontrolowanie drzwi:
		\begin{itemize}
		\item zamykanie,
		\item otwieranie,
		\item {\it programacja} drzwi,
		\end{itemize}
	\item sterowanie oświetleniem wewnętrznym i zewnętrznym,
	\item system zdalnej kontroli zwrotnic,
	\item działanie systemu GPS,
	\item symulacja systemu radiokomunikacji.
	\end{itemize}
\end{itemize}

\subsection{Otoczenie i ruch drogowy}
Jednym z kluczowych elementów szkolenia za pomocą symulatora tramwaju jest nabranie umiejętnośći
poruszania się w ruchu drogowym o wysokim natężeniu. Będzie to możliwe dzięki zastosowaniu
specjalnie do tego celu zaprojektowanego systemu sztucznej inteligencji. Umożliwi on kontrolę
poszczególnych pojazdów w ruchu drogowym, aby jak najbardziej zwiększyć realizm symulacji. Pojazdy
oprócz zaprogramowanych czynności będą mogły także wykonywać różne zdarzenia losowe.\\System
sztucznej inteligencji będzie także sterował pozostałymi tramwajami znajdującymi się na torowisku.

\subsection{Odwzorowanie miasta Wrocławia}
Istnieje możliwość zamówienia rozszerzonej wersji symulatora zawierającej wierne odwzorowanie układu
drogowego, torowiskowego, trakcyjnego i architektonicznego miasta Wrocławia, łącznie ze wszystkimi 
czterema zajezdniami tramwajowymi. Umożliwi ona pracę z symulatorem w warunkach jak najmniej 
odbiegających od rzeczywistego prowadzenia tramwaju {\it Škoda 16T}. Odwzorowanie to jest 
sprzedawane jako osobny produkt, gdyż wymaga ogromnego nakładu pracy.

\end{document}

