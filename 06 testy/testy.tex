\documentclass[12pt,a4paper]{article}
\usepackage{polski}
\usepackage[polish]{babel}
\usepackage{bbm}
\usepackage{bm}
\usepackage[T1]{fontenc}
\usepackage[utf8]{inputenc}
\usepackage[top=2cm, bottom=2cm, left=3cm, right=3cm]{geometry}
\usepackage{url}
\usepackage{graphics}
%\usepackage{graphicx}
\usepackage[pdftex]{graphicx}
\usepackage{float}
\usepackage{amsmath,amsthm}
\usepackage{enumitem}
\usepackage{epstopdf}
%\usepackage{indentfirst}
\usepackage[labelsep=period]{caption}
\setlist{nolistsep}


\AtBeginDocument{% Fragment zmieniający nazwy 'Rysunek' na 'Wykres' i 'Tablica' na 'Tabela'
        \renewcommand{\tablename}{Tabela}
        \renewcommand{\figurename}{Wykres}
}

\makeatletter
\newcommand{\linia}{\rule{\linewidth}{0.4mm}}


\floatstyle{plain}
\newfloat{image}{h!}{lop}
\floatname{image}{Rysunek}



\let \savenumberline \numberline
\def \numberline#1{\savenumberline{#1.}}


\renewcommand*{\@seccntformat}[1]{%
	  \csname the#1\endcsname
		    .\quad
}


\renewcommand{\maketitle}{\begin{titlepage}
		\vspace*{1cm}
    \begin{center}\small
    	Uniwersytet Wrocławski\\
    	Wydział Matematyki i Informatyki\\
    \end{center}
    \vspace{3cm}
    \noindent
    \linia
    \begin{center}
    	\LARGE{\textsc{\@title}}
         \end{center}
     \linia
    \begin{center}
    	\Large{Plan testowania}
         \end{center}
    \vspace{0.5cm}

    \begin{flushright}

    \begin{minipage}{5.5cm}

    	\small Autorzy:

    \normalsize {\@author} \par
    

    \end{minipage}
    \vspace{5cm}

     

     \end{flushright}

    \vspace*{\stretch{6}}

    \begin{center}

    \@date\\

    \end{center}

  \end{titlepage}%

}


\makeatother

\author{Jakub Stępniewicz (\textbf{233217})\\Rafał Maćkowski (\textbf{233170})\\Grupa {\bf I}}

\title{Symulator tramwaju\\ \small{Też możesz być motorniczym}}


\begin{document}
\maketitle
\tableofcontents
\vspace{5cm}
%	\begin{thebibliography}{9}
%	\bibitem{US} tz, W. Hill {\it Us}, Warszawa 2009.
%	\bibitem{MPK} \url{http://www.mpk.wroc.pl/}
%	\bibitem{SK} \url{http://www.skoda.cz/en/products/tramcars/tramcar-19-t/}
%	\bibitem{UE} \url{http://www.unrealengine.com/}
%	\bibitem{GO} \url{http://www.google.pl/#sclient=psy-ab&hl=pl&source=hp&q=%22symulator+tramwaju+skoda+16t%22&pbx=1&oq=%22symulator+tramwaju+skoda+16t}
%	\end{thebibliography}
\newpage
% 		Ok, najtrudniejsze za nami.		%
% 
\section{Wstęp}
Celem przeprowadzania procedur testowych jest wykrycie możliwie największej ilości błędów 
oprogramowania oraz wad wynikających z nieprawidłowej budowy sprzętu. Umożliwia to poprawienie 
konstrukcji symulatora oraz zmniejsza prawdopodobieństwo wystąpienia błędów po dostarczeniu go 
klientowi.

\section{Testowanie oprogramowania}
\subsection{Testy stabilności}
Testy stabilności polegają na próbie wykrycia błędów w programie, które mogłyby powodować brak 
odpowiedzi z jego strony lub konieczność zamknięcia procesu symulacji. Proces testowania polegałby 
na próbie wywołania wszystkich funkcji, składających się na oprogramowanie symulatora, w zależności 
od wartości, losowanej ze zbioru możliwych danych wejściowych dla danej funkcji. W razie wystąpienia 
błędów podczas przeprowadzania procedury, jesteśmy w stanie dokładnie sprecyzować, która funkcja 
powoduje nieprawidłowość oraz dla jakich argumentów. Powyższa metoda testowania stabilności nie 
pozwoli wykryć błędów wynikających z wprowadzenia niepoprawnych danych wejściowych jako argumentów 
funkcji.

\subsection{Testy publiczne}
Testy publiczne polegają na bezpłatnym udostępnieniu zmodyfikowanego pliku wykonywalnego w sieci 
Internet. Modyfikacje pliku umożliwiałyby przeprowadzenie symulacji bez użycia specjalnego kokpitu, a jedynie
używając klawiatury oraz 
myszy. Dodatkowo diody i wyświetlacze z konsoli byłyby widoczne na monitorze komputera. Plik byłby 
udostępniony jako gra komputerowa z możliwością wysyłania raportów o błędach generowanych przez 
program lub pisanych przez użytkownika.

\section{Badanie poprawności obwodów}
Badanie poprawności obwodów ma na celu wykrycie usterek związanych z niepoprawnym złożeniem lub 
zlutowaniem elementów składających się na symulator. Każdy obwód będzie badany przy pomocy 
mierników, aby sprawdzić przepływ prądu. Tam gdzie jest to możliwe, przeprowadzana będzie symulacja 
polegająca na przyłożeniu napięcia do styków urządzenia i sprawdzeniu czy urządzenie reaguje w 
sposób zgodny z oczekiwaniami.

\section{Testowanie sprzętu z zainstalowanym oprogramowaniem}
Testy wydajnościowe mają na celu znalezienie sytuacji w których program będzie generował obraz 
zauważalnie wolniej niż miałoby to miejsce w rzeczywistości. Na potrzeby testu przyjmujemy, że 
płynny obraz to wyświetlanie 30 klatek na sekundę. Aby umożliwić przeprowadzenie testów zostanie 
dopisana procedura badająca ilość klatek wyświetlanych na sekundę. W przypadku spadku tej wartości 
poniżej 30, w pliku stress\_test.txt zapisywane będą aktualne ustawienia programu oraz uzyskana 
liczba klatek na sekundę. Testy będą przeprowadzane dla różnych wartości parametrów symulacji.
\end{document}

